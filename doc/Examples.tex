%\documentstyle[12pt]{article}
%\documentstyle[twocolumn]{article}
\documentstyle{article}
%
% No Page Numbers!
%\pagestyle{empty}
%
%\raggedbottom
%\raggedright
\oddsidemargin   0.40in
\evensidemargin  0.40in
\leftmargin  0.0in
\rightmargin 0.0in
\textwidth   5.32in               % Originally 345pt
\topmargin   0.30in               % Originally 27pt
\headheight  0.1in               % Originally 12pt
\headsep     0.1in               % Originally 25pt
\textheight  8.0in               % Originally 528pt
\topskip     0.0in               % Originally same as BASELINESKIP
\footheight  0.1in               % Originally 12pt
\footskip    0.3in               % Originally 30pt
%
%\parindent 0.1in
%
%\setcounter{page}{0}      % try not to print the page number on the titlepage
%\roman{page}             % This makes no sense!
%\pagenumbering{roman}
%
\begin{document}
%\bibliographystyle{plain}
%
%\maketitle
%\newpage
%

\begin{center}
    {\Large {\bf A++/P++ -- Getting Started } } \\
\end{center}

{\footnotesize
\begin{tabbing}
xxx \= xx \= xxxxx \= xxxxx \= xxxxx \= xxxxx \= xxxxx \= xxxxx \= xxxxx \= xxxxx \= \kill

1   \> \#define BOUNDS\_CHECK \\
    \\
2   \> \#include "A++.h" \\
    \\
    \> void main () \\
    \> \{ \\
3   \>\>  int Array\_Size = 100; \\
    \\
    \>// Index Constructor examples \\
4   \>\> Index I ( 1 , Array\_Size-2, 1 ); \>\>\>\>\> // position=1, count=Array\_Size-2, stride=1 \\
5   \>\> Index J = I;                      \>\>\>\>\> // make an Index object J just like  I \\
6   \>\> Index K = I-1;                    \>\>\>\>\> // make an Index object K just like I-1 \\
7   \>\> Index L = -I;                     \>\>\>\>\> // make L like I but with negative stride \\
8   \>\> Index M = 5;                      \>\>\>\>\> // make Index object from integer index \\
9   \>\> Index N;                          \>\>\>\>\> // build an uninitialized Index object \\
10  \>\> N = I+1;                          \>\>\>\>\> // Index assignment to build N like offset of I \\
    \\
    \> // Array Constructor examples \\ 
11  \>\> doubleArray A1 (Array\_Size); \\ 
12  \>\> floatArray B1 (Array\_Size,Array\_Size); \\ 
13  \>\> doubleArray C1 (Array\_Size,Array\_Size,Array\_Size); \\ 
14  \>\> intArray D1 (Array\_Size,Array\_Size,Array\_Size,Array\_Size); \\ 
15  \>\> floatArray E1 = B1; \\ 
16  \>\> doubleArray F1 = B1(I-1,J); \\ 
    \\
17  \>\> double *Fortran\_Array\_Pointer = new double [Array\_Size+1][Array\_Size]; \\ 
18  \>\> doubleArray G (Fortran\_Array\_Pointer,Array\_Size,Array\_Size+1); \\ 
    \\
19  \>\> Arrays for use in examples below \\
20  \>\> doubleArray A (Array\_Size,Array\_Size); \\ 
21  \>\> doubleArray B (Array\_Size,Array\_Size); \\ 
22  \>\> doubleArray C (Array\_Size,Array\_Size); \\ 
23  \>\> doubleArray D (Array\_Size,Array\_Size); \\ 
24  \>\> double x = 42; \\ 
    \\
    \> // example of array-scalar assignment \\
25  \>\> A = x; \\
26  \>\> A (I) = x; \\
27  \>\> A (I-1) = x $*$ x; \\
    \\
    \> // examples of array-array assignment operations and use of Index objects \\
28  \>\> B = A; \\
29  \>\> B = C = D = A; \\
30  \>\> A (I,J) = B (J,J); \\
31  \>\> A (I-1,J) = B (I+1,J); \\
    \\
    \> // Scalar indexing \\
32  \>\> A (0,12) = x; \\
33  \>\> A (5,12) = A (0,12); \\
34  \>\> x = A (1,12) + B(0,12); \\
    \\
    \> // examples of array-array arithmitic operations \\
35  \>\> A = B + (C $*$ B - D) / A; \\
36  \>\> A (I,J) += B (I,J) / C (I,J); \\
37  \>\> A (I-1,J) $*=$ B (I+1,J); \\
    \\
    \> // examples of Jacobi relaxation (9-point stencile) \\
38  \>\> A (I,J) = ( A (I+1,J+1) + A (I,J+1) + A (I-1,J+1) + A (I+1,J) + \\ 
39  \>\>\>           A (I-1,J) + A (I+1,J-1) + A (I,J-1) + A (I-1,J-1) ) / 8.0; \\
    \\
    \> // examples of Jacobi relaxation (5-point stencile) \\
40  \>\> A (I,J) = ( A (I,J+1) + A (I,J-1) + A (I+1,J) + A (I-1,J) ) / 4.0; \\
    \\
    \> // more complex operations \\
41  \>\> B (I,J) = ( A (I-1,J-1) $*$ B (I+1,J+1) + C (I-1,J) $*$ D (I,J+1) -  \\ 
    \>\> \> \>       D (I,J) $*$ B (I,J) $*$ ( A (I,J) - B (I,J) ) ) / ( C (I,J) + D (I,J) ); \\
    \\
    \>// examples of relational operator \\
42  \>\> {\bf intArray} Mask = B $>=$ C; \\
43  \>\> Mask = !B; \\
44  \>\> Mask = !(B \&\& C) != (!B $\|$ !C); DeMorgan's Law \\
    \\
    \>// example of replace operator \\
45  \>\> A (I,J).replace ( B (I,J) $<=$ 0.001 , 0.001 ); \\ 
46  \>\> A (I,J).replace ( A (I,J) $<=$ C(I,J) , C(I,J) ); \\ 
    \\
    \>// examples of max function use \\
47  \>\> x = max (B); \\
48  \>\> A = max (B , C $*$ B); \\
49  \>\> A = max (B , C , A); \\
    \\
    \>// examples of miscellaneous function use \\
50  \>\> x = sum (B); \\
51  \>\> A = cos (B) $*$ sqrt (C); \\
52  \>\> B(I,J) = (cos (B) $*$ 2.0 )(I,J); \\
    \\
    \>// examples of changing bases of array objects \\
53  \>\> A.setBase (1);  Force A to have indexing similar to Fortran array \\
54  \>\> setGlobalBase (1); Set all future arrays to have Fortran like base of 1 \\
55  \>\> A.setBase (x); \\
56  \>\> A(x) = A(x+1); \\
57  \>\> A.setBase (x,0); \\
58  \>\> A.setBase (x*x,1); \\
    \\
    \>// examples of bases and bound access \\
59  \>\> Array\_Size = A.dimension(0); \\
60  \>\> printf ("Number of elements in A = \%d $\backslash$n",A.elementCount(); \\
61  \>\> for (int j = A.getBase(1); j $<=$ A.getBound(1); j++) \\
620 \>\>\> for (int i = A.getBase(0); i $<=$ A.getBound(0); i++) \\
63  \>\>\>\>  A(i,j) = foo (i,j); \\
    \\
    \>// examples of display functions \\
64  \>\> A (I,J).display("This is A (I,J)"); \\
65  \>\> A = B + (C $*$ D).display("This is C $*$ D in expresion A = B + (C $*$ D)"); \\
66  \>\> (A = B $*$ D).view("This is A = B $*$ D"); \\
67  \>\> A.view("This is A (same view as above)"); \\
    \\
    \>// 2 ways to pass array objects by reference \\
68  \>\> void foo ( {\bf const doubleArray} \& X ); \\
69  \>\> foo ( evaluate (A + B) ); \\
    \\
70  \>\> C = A + B; \\
71  \>\> foo ( C ); \\
    \\
    \>// passing array objects by value requires no special handling \\
72  \>\> void foobar ( {\bf const doubleArray} X ); \\
73  \>\> foobar ( A + B ); \\
    \\
    \>// examples of fill functions \\
74  \>\> A(I,J).fill(x); \\
    \\
75  \>\> printf ("PROGRAM TERMINATED NORMALLY $\backslash$n"); \\ 
    \> \} \\ 
\end{tabbing}

% End of font size scope
}

\end{document}














